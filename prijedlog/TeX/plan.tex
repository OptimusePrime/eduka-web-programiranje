\section{Plan radionice}\label{sec:plan-radionice}
Ciklus radionica bit će podijeljen na 4 radionice u trajanju od 3 školska sata (2h i 15 min).
Tijekom ciklusa, polaznici će naučiti tri osnovne tehnologije koje čine suvremeni Web: HTML, CSS i JavaScript, započevši s HTML-om.
Upoznati će s i s osnovama računalnih mreža, tj.\ na koji način sam Web funkcionira.

Polaznici će biti podučavani koristeći suvremene tehnike.
Između svakog školskog sata bit će kratak odmor 5--10 minuta.

\subsection{Detaljna razrada}\label{subsec:detaljna-razrada}

\begin{itemize}
    \item \textbf{Radionica 1 -- Uvod u web programiranje}
        \begin{enumerate}
            \item Prvi sat
                \begin{itemize}
                    \item Upoznavanje i uvod -- 10 min
                    \item Kako radi Internet?\ -- 10 min
                    \item Kako rade web stranice?\ -- 10 min
                    \item Uvod u HTML -- 15 min
                \end{itemize}
            \item Drugi sat
                \begin{itemize}
                    \item Postavljanje radnog okruženja -- 15 min
                    \item HTML tagovi -- 10 min
                    \item Najčešći HTML tagovi -- 20 min
                    \item Prihvaćanje korisničkih podataka -- 10 min
                \end{itemize}
            \item Treći sat
                \begin{itemize}
                    \item Head i Meta tagovi -- 7 min
                    \item Organizacija HTML-a -- 20--25 min
                    \item Uljepšavanje stranica i uvod u CSS -- 15 min
                \end{itemize}
        \end{enumerate}
    \item \textbf{Radionica 2 -- CSS i uvod u JavaScript}
        \begin{enumerate}
            \item Prvi sat
                \begin{itemize}
                    \item Ponavljanje -- 10 min
                    \item Selektori -- 15 min
                    \item Terminologija -- 5 min
                    \item Kaskada i nasljeđivanje -- 15 min
                \end{itemize}
            \item Drugi sat
                \begin{itemize}
                    \item Model \("\)kutije\("\) -- 15 min
                    \item CSS Flexbox -- 15--20 min
                    \item CSS Grid -- 15--20 min
                \end{itemize}
            \item Treći sat
                \begin{itemize}
                    \item CSS i HTML projekt --\ 30 min
                    \item Uvod u JavaScript -- 15 min
                \end{itemize}
        \end{enumerate}
    \item \textbf{Radionica 3 -- Osnove JavaScripta}
        \begin{enumerate}
            \item Prvi sat
                \begin{itemize}
                    \item Ponavljanje -- 15 min
                    \item Osnove JavaScripta -- 20 min
                    \item Skalarni tipovi podataka -- 15 min
                \end{itemize}
            \item Drugi sat
                \begin{itemize}
                    \item Upravljanje tijeka programa -- 15 min
                    \item Petlje -- 15 min
                    \item Funkcije -- 15 min
                \end{itemize}
            \item Treći sat
                \begin{itemize}
                    \item Polje vidljivosti -- 15 min
                    \item Standardna bibliotetka -- 20 min
                    \item Nizovi -- 15 min
                \end{itemize}
        \end{enumerate}
    \item \textbf{Radionica 4 -- JavaScript na Webu}
        \begin{enumerate}
            \item Prvi sat
                \begin{itemize}
                    \item Ponavljanje -- 25 min
                    \item Objekti -- 20 min
                \end{itemize}
            \item Drugi sat
                \begin{itemize}
                    \item Upravljanje HTML-om i CSS-om -- 35 min
                    \item Vježba -- 10 min
                \end{itemize}
            \item Treći sat
                \begin{itemize}
                    \item Izrada igrice Wordle -- 35--45 min
                    \item \("\)Kuda dalje?\("\) i druga pitanja
                \end{itemize}
        \end{enumerate}
\end{itemize}

Ciklus ćemo započeti s upoznavanjem s predavačem, ali i upoznavanjem učenika.
Slijedit će kratko predavanje o tome što je Internet (razlika između Weba i Interneta i sl.), kako radi (TCP/IP, DNS, itd.) te ključni dijelovi njegove povijest (ARPANet, razvoj WWW-a i sl.).
Tijekom cijelog ciklusa učenici će moći postavljati pitanja.
Ako učenici neće imati više pitanja, pojasnit ćemo detaljnije što čini web stranice te kako se one razvijaju (HTML, CSS, JavaScript, suvremene web aplikacije, itd).
Nakon toga ćemo započeti uvodno predavanja o HTML-u te napraviti jednostavnu stranicu s njime u Bloku za pisanje.

Kako programiranje u Bloku za pisanje (Notepad) nije najbolje iskustvo, na početku drugoga sata učenike ćemo naučiti kako postaviti njihovo vlastito radno okruženja koristeći Visual Studio Code (\textbf{VSC}), najpopularnije integrirano razvojno okruženje (eng. \textbf{IDE}) za web programiranje.
VSC ima razne značajke pogodne za brzi razvoj programa.
Pokriti ćemo najčešće korištene HTML tagove i elemente te objasniti kada koristiti koji.
U drugom djelu drugoga sata pojasnit ćemo kako web stranice prihvaćaju unošenje korisničkih podataka putem obrazaca.

Radionicu ćemo završiti s predavanjem o tagovima koji govore web preglednicima i pretraživačima podatke o samoj stranici (naslov i slično).
Pretkraj bit će napomenut problem poboljšavanja izgleda web stranica (isključivo korištenje HTML-a stvara web stranice s vrlo lošim dizajnom).
Taj će razgovor dovesti do kratkog uvoda u CSS, jezika koji se na Webu koristi za određivanje izgleda HTML elemenata.
Kroz cijeli ciklus pokušat ćemo koristiti što više praktičnih prikaza, tako da u svakom dijelu ciklusa napravimo nekakav praktični dio stranice.

Druga radionica započeti će s ponavljanjem HTML-a kroz kratku vježbu.
Nastavit ćemo s detaljnim pregledom CSS selektora i ostalih osnovnih značajka CSS-a.
Drugi sat bit će zauzet s dvije naprednije značajke CSS-a koje su vrlo bitne za suvremeno web programiranje.

Većina trećeg sata bit će zauzeta duljim projektom u kojem ćemo napraviti jednostavnu web stranicu uz pomoć HTML-a i CSS-a.
Iako jednostavna, pomoći će učenicima u isto vrijeme koristiti obje naučene tehnologije, HTML i CSS, u praktičnom kontekstu.
Ako će vrijeme dozvoliti, ostatak sata bit će dodijeljeno uvodom u osnove JavaScripta.

Ostale dvije radionice bit će većinski zauzete JavaScriptom, koji je i zapravo glavni jezik Weba.
Kada je kod moguće, JavaScript će biti uspoređen s Pythonom, jezik s kojim su učenici već upoznati.
To bi trebalo poboljšati i ubrzati njihovo razumijevanje JavaScripta.
U zadnjem satu, bit će prikazana izrada vrlo popularne, ali i jednostavne, internetske igrice, Wordle.
Programiranje te igrice povezat će sve naučeno u zadnje četiri radionice i privesti kraju ovaj ciklus radionica.

Ako vrijeme dozvoli, zadnji dio će sadržavati savjete za daljnje korake za one koji žele nastaviti učiti (web) programiranje.
