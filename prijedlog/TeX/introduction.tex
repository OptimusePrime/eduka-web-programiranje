% TeX root = ../main.tex

\section{Motivacija}\label{sec:introduction}
Računalna tehnologija ima, u usporedi s ostalima, vrlo kratku povijest.
Suvremeno računarstvo ima svoje korijene u ranom i srednjem 20.\ stoljeću.
Međutim, računala tada su bila nepraktična, zauzimala su previše prostora i bila prekomplicirana za svakodnevno korištenje.

Sve se započelo mijenjati krajem prošloga stoljeća kada su računala krenula poprimati svoji suvremeni oblik.
Korisnička sučelja postala su pristupačnija i njihova procesorska snaga postala je dovoljna velika da omogući korištenje u svakodnevici.
Društvene promjene pokrenute razvojem računalne tehnologije bile su ubrzane razvojem sustav znan kao World Wide Web (\textbf{WWW} ili samo Web).
Iako su razni oblici Interneta postojali već od sedamdesetih godina prošlog stoljeća, Web je bio onaj koji je popularizirao Internet.
U isto vrijeme, samo računalo je postajalo pristupačno prosječnom građaninu.

Danas, informatička tehnologija podupire gotovo svaki dio suvremenog društva, od sustava za pročišćavanje vode do zrakoplova i nuklearnih reaktora.
Informatička pismenost jedna je od najbitnijih vještina za posjedovati u današnjici.
S tim ciljem, ova radionica namjerava omogućiti polaznicima da steknu bitno znanje o Webu.
Učenjem osnovnog web programiranja, učenici će steći dublje razumijevanje tehnologija koje čine suvremeni Internet.
To će znanje razbiti mnogo čestih mitova o Webu i širem računarstvom.
Mnogi od njih možda će i postati profesionalni informatičari u budućnosti.
Popularizacija računarstva trebala bi biti bitna informatičarima, bez novih informatičara nema informatike.


