% --- LaTeX Research Paper Template - S. Venkatraman ---

% --- Set document class and font size ---

\documentclass[letterpaper, 11pt]{article}

% --- Package imports ---

\usepackage[croatian]{babel}
\usepackage{
    amsmath, amsthm, amssymb, mathtools, dsfont, units,
    graphicx, wrapfig, subfig, float,
    listings, color, inconsolata, pythonhighlight,
    fancyhdr, sectsty, hyperref, enumerate, enumitem }

% lipsum is just for generating placeholder text and can be removed
\usepackage{lipsum}

% --- Page layout settings ---

% Set page margins
\usepackage[left=1.15in, right=1.15in, top=1.2in, bottom=1in, headsep=.25in]{geometry}

% Anchor footnotes to the bottom of the page
\usepackage[bottom]{footmisc}

% Set line spacing
\renewcommand{\baselinestretch}{1.12}

% Set spacing between paragraphs
\setlength{\parskip}{1.5mm}

% Allow multi-line equations to break onto the next page
\allowdisplaybreaks

% --- Page formatting settings ---

% Set font sizes for section and subsection titles
\sectionfont{\fontsize{13}{15}\selectfont}
\subsectionfont{\fontsize{11}{15}\selectfont}

% Set reference section title font size the same as the section font size
\renewcommand{\refname}{\fontsize{13}{15}\selectfont\bf{References}}

% Set link colors for labeled items and citations (blue) and citations (red)
\hypersetup{colorlinks=true, linkcolor=blue, citecolor=blue}

% Abstract settings
\renewenvironment{abstract}
{\small\begin{quote}
           \noindent \par{\sc \abstractname.}}
           {\noindent
\end{quote}}

% --- Settings for printing computer code ---

% Define colors for green text (comments), grey text (line numbers),
% green frame around code, 
\definecolor{greenText}{rgb}{0.5, 0.7, 0.5}
\definecolor{greyText}{rgb}{0.5, 0.5, 0.5}
\definecolor{codeFrame}{rgb}{0.5, 0.7, 0.5}

% Define code settings
\lstdefinestyle{code} {
    frame=single, rulecolor=\color{codeFrame},            % Include a green frame around the code
    numbers=left,                                         % Include line numbers
    numbersep=8pt,                                        % Add space between line numbers and frame
    numberstyle=\tiny\color{greyText},                    % Line number font size (tiny) and color (grey)
    commentstyle=\color{greenText},                       % Put comments in green text
    basicstyle=\linespread{1.1}\ttfamily\footnotesize,    % Set code line spacing
    keywordstyle=\ttfamily\footnotesize,                  % No special formatting for keywords
    showstringspaces=false,                               % No marks for spaces
    xleftmargin=1.95em,                                   % Align code frame with main text
    framexleftmargin=1.6em,                               % Extend frame left margin to include line numbers
    breaklines=true,                                      % Wrap long lines of code
    postbreak=\mbox{\textcolor{greenText}{$\hookrightarrow$}\space} % Mark wrapped lines with an arrow
}

% Set all code listings to be styled with the above settings
\lstset{style=code}

% --- Math/Statistics commands ---

% Add a reference number to a single line of a multi-line equation
% Usage: "\numberthis\label{labelNameHere}" in an align or gather environment
\newcommand\numberthis{\addtocounter{equation}{1}\tag{\theequation}}

% Shortcut for bold text in math mode, e.g. $\b{X}$
\let\b\mathbf

% Shortcut for bold Greek letters, e.g. $\bg{\beta}$
\let\bg\boldsymbol

% Shortcut for calligraphic script, e.g. %\mc{M}$
\let\mc\mathcal

% \mathscr{(letter here)} is sometimes used to denote vector spaces
\usepackage[mathscr]{euscript}
\usepackage{atveryend}
\usepackage{amstex}
\usepackage{newpxmath}

% Convergence: right arrow with optional text on top
% E.g. $\converge[w]$ for weak convergence
\newcommand{\converge}[1][]{\xrightarrow{#1}}

% Normal distribution: arguments are the mean and variance
% E.g. $\normal{\mu}{\sigma}$
\newcommand{\normal}[2]{\mathcal{N}\left(#1,#2\right)}

% Uniform distribution: arguments are the left and right endpoints
% E.g. $\unif{0}{1}$
\newcommand{\unif}[2]{\text{Uniform}(#1,#2)}

% Independent and identically distributed random variables
% E.g. $ X_1,...,X_n \iid \normal{0}{1}$
\newcommand{\iid}{\stackrel{\smash{\text{iid}}}{\sim}}

% Equality: equals sign with optional text on top
% E.g. $X \equals[d] Y$ for equality in distribution
\newcommand{\equals}[1][]{\stackrel{\smash{#1}}{=}}

% Math mode symbols for common sets and spaces. Example usage: $\R$
\newcommand{\R}{\mathbb{R}}   % Real numbers
\newcommand{\C}{\mathbb{C}}   % Complex numbers
\newcommand{\Q}{\mathbb{Q}}   % Rational numbers
\newcommand{\Z}{\mathbb{Z}}   % Integers
\newcommand{\N}{\mathbb{N}}   % Natural numbers
\newcommand{\F}{\mathcal{F}}  % Calligraphic F for a sigma algebra
\newcommand{\El}{\mathcal{L}} % Calligraphic L, e.g. for L^p spaces

% Math mode symbols for probability
\newcommand{\pr}{\mathbb{P}}    % Probability measure
\newcommand{\E}{\mathbb{E}}     % Expectation, e.g. $\E(X)$
\newcommand{\var}{\text{Var}}   % Variance, e.g. $\var(X)$
\newcommand{\cov}{\text{Cov}}   % Covariance, e.g. $\cov(X,Y)$
\newcommand{\corr}{\text{Corr}} % Correlation, e.g. $\corr(X,Y)$
\newcommand{\B}{\mathcal{B}}    % Borel sigma-algebra

% Other miscellaneous symbols
\newcommand{\tth}{\text{th}}    % Non-italicized 'th', e.g. $n^\tth$
\newcommand{\Oh}{\mathcal{O}}    % Big-O notation, e.g. $\O(n)$
\newcommand{\1}{\mathds{1}}        % Indicator function, e.g. $\1_A$

% Additional commands for math mode
\DeclareMathOperator*{\argmax}{argmax}    % Argmax, e.g. $\argmax_{x\in[0,1]} f(x)$
\DeclareMathOperator*{\argmin}{argmin}    % Argmin, e.g. $\argmin_{x\in[0,1]} f(x)$
\DeclareMathOperator*{\spann}{Span}       % Span, e.g. $\spann\{X_1,...,X_n\}$
\DeclareMathOperator*{\bias}{Bias}        % Bias, e.g. $\bias(\hat\theta)$
\DeclareMathOperator*{\ran}{ran}          % Range of an operator, e.g. $\ran(T) 
\DeclareMathOperator*{\dv}{d\!}           % Non-italicized 'with respect to', e.g. $\int f(x) \dv x$
\DeclareMathOperator*{\diag}{diag}        % Diagonal of a matrix, e.g. $\diag(M)$
\DeclareMathOperator*{\trace}{trace}      % Trace of a matrix, e.g. $\trace(M)$

% Numbered theorem, lemma, etc. settings - e.g., a definition, lemma, and theorem appearing in that 
% order in Section 2 will be numbered Definition 2.1, Lemma 2.2, Theorem 2.3. 
% Example usage: \begin{theorem}[Name of theorem] Theorem statement \end{theorem}
\theoremstyle{definition}
\newtheorem{theorem}{Theorem}[section]
\newtheorem{proposition}[theorem]{Proposition}
\newtheorem{lemma}[theorem]{Lemma}
\newtheorem{corollary}[theorem]{Corollary}
\newtheorem{definition}[theorem]{Definition}
\newtheorem{example}[theorem]{Example}
\newtheorem{remark}[theorem]{Remark}

% --- Left/right header text (to appear on every page) ---

% Do not include a line under header or above footer
\pagestyle{fancy}
\renewcommand{\footrulewidth}{0pt}
\renewcommand{\headrulewidth}{0pt}

% Center header text: short paper title
\chead{\MakeUppercase{\scriptsize Prijedlog za radionicu web programiranja}}

% Empty left and right header text
\lhead{}\rhead{}

% --- Document starts here ---

\begin{document}

% --- Title, author, date ---

    \title{\normalsize\MakeUppercase{\bfseries
    Prijedlog za radionicu iz osnova web programiranja za učenike sedmih i osmih razreda osnovne škole}}
    \author{\small\MakeUppercase{
        Karlo Vizec}}
    \date{\footnotesize\MakeUppercase\today}
    \maketitle
    \vspace{-1cm}

% --- Main content: import sections as subfiles ---

    % TeX root = ../main.tex

\begin{abstract}
    Popularizacija programiranja i ostalih STEM područja opće je prihvaćeno kao bitno za znanost i šire društvo.
    Radionice programiranja za mlade jedan su od prepoznatih načina za ostvarivanje tog cilja.
    Web programiranje jedno je od najpopularnijih oblika programiranja te jedno od najlakših za započeti.
    Usto, tehnologije koje se koriste za izradu web aplikacija mogu se koristiti i za razvoj raznih drugih programa, što omogućuje lakši pristup drugim područjima programskog inženjerstva.
    Učenje programiranje (u ranoj dobi) ojačava sposobnosti rješavanja problema i analitičkog, kritičkog razmišljanja među mladima.
\end{abstract}
    % TeX root = ../main.tex

\section{Motivacija}\label{sec:introduction}
Računalna tehnologija ima, u usporedi s ostalima, vrlo kratku povijest.
Suvremeno računarstvo ima svoje korijene u ranom i srednjem 20.\ stoljeću.
Međutim, računala tada su bila nepraktična, zauzimala su previše prostora i bila prekomplicirana za svakodnevno korištenje.

Sve se započelo mijenjati krajem prošloga stoljeća kada su računala krenula poprimati svoji suvremeni oblik.
Korisnička sučelja postala su pristupačnija i njihova procesorska snaga postala je dovoljna velika da omogući korištenje u svakodnevici.
Društvene promjene pokrenute razvojem računalne tehnologije bile su ubrzane razvojem sustav znan kao World Wide Web (\textbf{WWW} ili samo Web).
Iako su razni oblici Interneta postojali već od sedamdesetih godina prošlog stoljeća, Web je bio onaj koji je popularizirao Internet.
U isto vrijeme, samo računalo je postajalo pristupačno prosječnom građaninu.

Danas, informatička tehnologija podupire gotovo svaki dio suvremenog društva, od sustava za pročišćavanje vode do zrakoplova i nuklearnih reaktora.
Informatička pismenost jedna je od najbitnijih vještina za posjedovati u današnjici.
S tim ciljem, ova radionica namjerava omogućiti polaznicima da steknu bitno znanje o Webu.
Učenjem osnovnog web programiranja, učenici će steći dublje razumijevanje tehnologija koje čine suvremeni Internet.
To će znanje razbiti mnogo čestih mitova o Webu i širem računarstvom.
Mnogi od njih možda će i postati profesionalni informatičari u budućnosti.
Popularizacija računarstva trebala bi biti bitna informatičarima, bez novih informatičara nema informatike.



    \section{Plan radionice}\label{sec:plan-radionice}
Ciklus radionica bit će podijeljen na 4 radionice u trajanju od 3 školska sata (2h i 15 min).
Tijekom ciklusa, polaznici će naučiti tri osnovne tehnologije koje čine suvremeni Web: HTML, CSS i JavaScript, započevši s HTML-om.
Upoznati će s i s osnovama računalnih mreža, tj.\ na koji način sam Web funkcionira.

Polaznici će biti podučavani koristeći suvremene tehnike.
Između svakog školskog sata bit će kratak odmor 5--10 minuta.

\subsection{Detaljna razrada}\label{subsec:detaljna-razrada}

\begin{itemize}
    \item \textbf{Radionica 1 -- Uvod u web programiranje}
        \begin{enumerate}
            \item Prvi sat
                \begin{itemize}
                    \item Upoznavanje i uvod -- 10 min
                    \item Kako radi Internet?\ -- 10 min
                    \item Kako rade web stranice?\ -- 10 min
                    \item Uvod u HTML -- 15 min
                \end{itemize}
            \item Drugi sat
                \begin{itemize}
                    \item Postavljanje radnog okruženja -- 15 min
                    \item HTML tagovi -- 10 min
                    \item Najčešći HTML tagovi -- 20 min
                    \item Prihvaćanje korisničkih podataka -- 10 min
                \end{itemize}
            \item Treći sat
                \begin{itemize}
                    \item Head i Meta tagovi -- 7 min
                    \item Organizacija HTML-a -- 20--25 min
                    \item Uljepšavanje stranica i uvod u CSS -- 15 min
                \end{itemize}
        \end{enumerate}
    \item \textbf{Radionica 2 -- CSS i uvod u JavaScript}
        \begin{enumerate}
            \item Prvi sat
                \begin{itemize}
                    \item Ponavljanje -- 10 min
                    \item Selektori -- 15 min
                    \item Terminologija -- 5 min
                    \item Kaskada i nasljeđivanje -- 15 min
                \end{itemize}
            \item Drugi sat
                \begin{itemize}
                    \item Model \("\)kutije\("\) -- 15 min
                    \item CSS Flexbox -- 15--20 min
                    \item CSS Grid -- 15--20 min
                \end{itemize}
            \item Treći sat
                \begin{itemize}
                    \item CSS i HTML projekt --\ 30 min
                    \item Uvod u JavaScript -- 15 min
                \end{itemize}
        \end{enumerate}
    \item \textbf{Radionica 3 -- Osnove JavaScripta}
        \begin{enumerate}
            \item Prvi sat
                \begin{itemize}
                    \item Ponavljanje -- 15 min
                    \item Osnove JavaScripta -- 20 min
                    \item Skalarni tipovi podataka -- 15 min
                \end{itemize}
            \item Drugi sat
                \begin{itemize}
                    \item Upravljanje tijeka programa -- 15 min
                    \item Petlje -- 15 min
                    \item Funkcije -- 15 min
                \end{itemize}
            \item Treći sat
                \begin{itemize}
                    \item Polje vidljivosti -- 15 min
                    \item Standardna bibliotetka -- 20 min
                    \item Nizovi -- 15 min
                \end{itemize}
        \end{enumerate}
    \item \textbf{Radionica 4 -- JavaScript na Webu}
        \begin{enumerate}
            \item Prvi sat
                \begin{itemize}
                    \item Ponavljanje -- 25 min
                    \item Objekti -- 20 min
                \end{itemize}
            \item Drugi sat
                \begin{itemize}
                    \item Upravljanje HTML-om i CSS-om -- 35 min
                    \item Vježba -- 10 min
                \end{itemize}
            \item Treći sat
                \begin{itemize}
                    \item Izrada igrice Wordle -- 35--45 min
                    \item \("\)Kuda dalje?\("\) i druga pitanja
                \end{itemize}
        \end{enumerate}
\end{itemize}

Ciklus ćemo započeti s upoznavanjem s predavačem, ali i upoznavanjem učenika.
Slijedit će kratko predavanje o tome što je Internet (razlika između Weba i Interneta i sl.), kako radi (TCP/IP, DNS, itd.) te ključni dijelovi njegove povijest (ARPANet, razvoj WWW-a i sl.).
Tijekom cijelog ciklusa učenici će moći postavljati pitanja.
Ako učenici neće imati više pitanja, pojasnit ćemo detaljnije što čini web stranice te kako se one razvijaju (HTML, CSS, JavaScript, suvremene web aplikacije, itd).
Nakon toga ćemo započeti uvodno predavanja o HTML-u te napraviti jednostavnu stranicu s njime u Bloku za pisanje.

Kako programiranje u Bloku za pisanje (Notepad) nije najbolje iskustvo, na početku drugoga sata učenike ćemo naučiti kako postaviti njihovo vlastito radno okruženja koristeći Visual Studio Code (\textbf{VSC}), najpopularnije integrirano razvojno okruženje (eng. \textbf{IDE}) za web programiranje.
VSC ima razne značajke pogodne za brzi razvoj programa.
Pokriti ćemo najčešće korištene HTML tagove i elemente te objasniti kada koristiti koji.
U drugom djelu drugoga sata pojasnit ćemo kako web stranice prihvaćaju unošenje korisničkih podataka putem obrazaca.

Radionicu ćemo završiti s predavanjem o tagovima koji govore web preglednicima i pretraživačima podatke o samoj stranici (naslov i slično).
Pretkraj bit će napomenut problem poboljšavanja izgleda web stranica (isključivo korištenje HTML-a stvara web stranice s vrlo lošim dizajnom).
Taj će razgovor dovesti do kratkog uvoda u CSS, jezika koji se na Webu koristi za određivanje izgleda HTML elemenata.
Kroz cijeli ciklus pokušat ćemo koristiti što više praktičnih prikaza, tako da u svakom dijelu ciklusa napravimo nekakav praktični dio stranice.

Druga radionica započeti će s ponavljanjem HTML-a kroz kratku vježbu.
Nastavit ćemo s detaljnim pregledom CSS selektora i ostalih osnovnih značajka CSS-a.
Drugi sat bit će zauzet s dvije naprednije značajke CSS-a koje su vrlo bitne za suvremeno web programiranje.

Većina trećeg sata bit će zauzeta duljim projektom u kojem ćemo napraviti jednostavnu web stranicu uz pomoć HTML-a i CSS-a.
Iako jednostavna, pomoći će učenicima u isto vrijeme koristiti obje naučene tehnologije, HTML i CSS, u praktičnom kontekstu.
Ako će vrijeme dozvoliti, ostatak sata bit će dodijeljeno uvodom u osnove JavaScripta.

Ostale dvije radionice bit će većinski zauzete JavaScriptom, koji je i zapravo glavni jezik Weba.
Kada je kod moguće, JavaScript će biti uspoređen s Pythonom, jezik s kojim su učenici već upoznati.
To bi trebalo poboljšati i ubrzati njihovo razumijevanje JavaScripta.
U zadnjem satu, bit će prikazana izrada vrlo popularne, ali i jednostavne, internetske igrice, Wordle.
Programiranje te igrice povezat će sve naučeno u zadnje četiri radionice i privesti kraju ovaj ciklus radionica.

Ako vrijeme dozvoli, zadnji dio će sadržavati savjete za daljnje korake za one koji žele nastaviti učiti (web) programiranje.

    \input{TeX/tehnike}

% --- Bibliography ---

    \bibliography{TeX/references}
    \bibliographystyle{apalike2}

% --- Document ends here ---

\end{document}
