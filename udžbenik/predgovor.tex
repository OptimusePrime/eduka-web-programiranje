\chapter*{Predgovor}\label{ch:predgovor}

Računala prošlosti bila su ogromna.
Tijekom sredine prošlog stoljeća zauzimala su cijele prostorije, a bila sporija od najslabijeg suvremenog mobitela.
Imala su vrlo malu memoriju te je njihova funkcija bila ograničena procesorskom snagom koja je bila niska.
Međutim, najbitnija činjenica za nas je da nisu bila povezana.

Recimo da želite instalirati Windows 3.0 na vaše novo IBM osobno računalo tijekom 1990.
Vaš prvi korak je da odete do najbližeg dućana s računalima i kupite fizički primjerak Windowsa na disketi.
Kada stignete doma, tu disketu bi ubacili u vaše računalo i instalirali Windows.
Tada Internet nije postojao, niste mogli samo otvoriti microsoft.com i preuzeti najnoviju verziju.
Muzika se prenosila pomoću disketa, i kasnije CD-a i DVD-a, a videi su bili vrlo rijetki na računalima.

Danas je Internet jedan od najvažnijih dijelova našeg života, postao je novi način na koji prikupljamo informacije i razmišljamo.
Jedna rečenica objavljena na Twitteru od strane neke poznate osobe može imati ogroman utjecaj.
Zbog toga računala i mobiteli postaju sve više samo platforme za web preglednike.
Razne tvrtke to primjećuju i prilagođavaju svoji model poslovanja.
Nove tehnologije omogućuje da sve više i više aplikacija i programa postanu web stranice (web aplikacije) bez potrebe preuzimanja.

Programeri nisu slijepi na te promjene, a to se lako može primijetiti u činjenici da su jezici za web programiranje, načelno HTML, CSS i JavaScript, među najpopularnijim.
Štoviše, prema 2023.\ anketi za programere StackOverflowa, JavaScript je \textit{najpopularniji} programski jezik, a drugi najpopularniji su HTML i CSS.

Ovaj udžbenik sam napisao s ciljem otvaranja svijet informatike i programiranja srednjoškolcima i naprednim osnovnoškolcima.
Namijenjen je primarno za fakultativnu nastavu i/ili izvannastavne aktivnosti iz informatike i programiranja.
U udžbeniku ima dovoljno materijala za ~35 školskih sati godišnje, ovisno o opsežnosti nastave.
Nadam se da će ovaj udžbenik pridonijeti smanjivanju praga za učenje programiranja učenicima.

\medskip
\begin {flushright}
    \textit{Karlo Vizec, listopad 2023.}
\end {flushright}

